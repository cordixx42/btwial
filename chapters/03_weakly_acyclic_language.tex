\chapter{Weakly Acyclic Languages}\label{chapter:weakly_acyclic_language}

This chapter introduces a new class of a formal language, which we call weakly acyclic. 

\section{Weakly Acyclic DFAs}\label{sec:weakly_acyclic_dfa}

\begin{definition}[\cite{blondin_24}]
Let $A = (Q, \Sigma, q_{0},\delta,F)$ be a DFA and let $\alpha(w)$ be the set of letter, which occur within in word $w$. DFA $A$ is weakly acyclic, if $\delta(q,w) = q$ implies $\delta(q,a) = q$ for every $a \in \alpha(w)$
\end{definition}

The following definitions are equivalently expressing, that a DFA $(Q, \Sigma, q_{0},\delta,F)$ is \textit{weakly acyclic}:
\begin{itemize}[--,noitemsep]
	\item the binary relation $\precsim $  over $Q \times Q$ with $q \precsim q'$ if $\delta(q,w) = q'$ is a partial order
	\item each strongly connected component of underlying directed graph contains a single state 
	\item the underlying directed graph does not contain any simple cycle except from self-loops
\end{itemize}

\todo{add figure of wa dfa as example}

\begin{lemma}\label{lem:minimal}
Let $A$ be a weakly acyclic DFA. The minimal DFA that accepts L$(A)$ is also weakly acyclic. 
\end{lemma}

\begin{proof}
See {\cite[Proposition~4]{blondin_24}}
\end{proof}


\section{Other Representations}

As with regular languages, there are equivalent ways to represent weakly acyclic languages. Let $\alpha(w)$ be the set of letter, which occur within in word $w$. 
Blondin et al. have shown, that weakly acyclic DFAs, weakly acyclic NFAs and weakly acyclic expressions represent the same class of weakly acyclic languages \cite{blondin_24}. 


%It can further be shown, that if a weakly acyclic DFA $A$ accepts the language $L(A)$, then the minimal automata for this language is also weakly acyclic \cite{blondin_24}. 

\subsection{Weakly Acyclic NFAs}
An NFA $(Q, \Sigma, q_{0},\delta,F)$ is \textit{weakly acyclic} if 
\begin{itemize}[--,noitemsep]
	\item $q \in \delta(q,w)$ implies $\delta(q,a) = {q}$ for every $a \in \alpha(w)$
	\item the underlying directed graph does not contain any simple cycle except from self-loops and non-determinism with a letter $a$ can only appear from a state with no self-loop of $a$.
\end{itemize}

\subsection{Weakly Acyclic Expressions}\label{sec:weakly_acyclic_regex}
The class of weakly acyclic languages can also be characterized by \textit{weakly acyclic expressions} of the following form
\begin{equation*}
r \Coloneqq \emptyset \mid \Gamma^{*} \mid \Lambda^{*} a r \mid r + r \quad \text{where} \quad  \Gamma, \Lambda \subseteq \Sigma \quad \textrm{and} \quad a \in \Sigma \setminus \Lambda
\end{equation*}

% TODO: propositions proving the equivalence of dfa nfa and regex in supplementary materials

\section{Properties}
\subsection{Position in Language Hierarchy}
The weakly acyclic language lie strictly in between finite and regular languages. 
They are within the regular languages, as they can be characterized with DFAs, NFAs and regular expressions. Furthermore, every finite language is weakly acyclic, as only a self loop in the trap state is necessary to represent a finite set of words.
\todo{strictness with examples}
\todo{hierarchy figure}


\subsection{Closure Properties}
Weakly acyclic languages are closed under union, intersection and complementation, but not under concatenation or Kleene star. 
\par 
As weakly acyclic languages can be described by DFAs as in \autoref{sec:weakly_acyclic_dfa}, complementing a DFA still preserves its weakly acyclic structure. Union is already present in the definition of weakly acyclic expressions in \autoref{sec:weakly_acyclic_regex}. Therefore, the languages are also closed under intersection, which can be expressed with the combination of union and complementation. 
\par
Weakly acyclic languages are not closed under concatenation. The expressions $\bm{(a + b)^{*}}$ and $\bm{b}$ are weakly acyclic by themselves. The language of their concatenation $\bm{(a + b)^{*}b}$ is depicted in a minimal DFA in \autoref{fig:not-closed-concat}.

\begin{figure}[t]
\centering 
	\begin{subfigure}{.30\textwidth}	
		\centering 
        \begin{tikzpicture}
    	\node[state, initial] (q1) {$q_{1}$};
    	\node[state, accepting, right of=q1] (q2) {$q_2$};
    	\draw   (q1) edge[loop above] node{$\bm{a}$} (q1)
            	(q1) edge[bend left, above] node{$\bm{b}$} (q2)
            	(q2) edge[loop above] node{$\bm{b}$} (q2)
            	(q2) edge[bend left, below] node{$\bm{a}$} (q1);
    	\end{tikzpicture}
    	\centering 
    	\caption{Minimal DFA of \\ $L(\bm{(a + b)^{*}b})$}\label{fig:not-closed-concat}
    \end{subfigure}
    \begin{subfigure}{.60\textwidth}
	\centering 
        \begin{tikzpicture}
    	\node[state, initial, accepting] (q1) {$q_1$};
    	\node[state, right of=q1] (q2) {$q_2$};
    	\node[state, right of=q2] (q3) {$q_3$};
    	\draw   (q1) edge[below] node{$\bm{a}$} (q2)
            	(q1) edge[bend right, below] node{$\bm{b}$} (q3)
            	(q2) edge[bend right, above] node{$\bm{b}$} (q1)
            	(q2) edge[below] node{$\bm{a}$} (q3)
            	(q3) edge[loop above] node{$\bm{a},\bm{b}$} (q3);
    	\end{tikzpicture}
    	\caption{Minimal DFA of $L(\bm{(ab)^{*}})$}\label{fig:not-closed-kleene}
    \end{subfigure}
     \caption{Not weakly acyclic DFA}
\end{figure}

By Lemma~\autoref{lem:minimal} the minimal automaton of weakly acyclic language remains weakly acyclic. As this automaton contains a cycle of length 2, this language is not weakly acyclic. The same argumentation can be made for the Kleene star closure, with $\bm{ab}$ being a weakly acyclic language, but $\bm{(ab)^{*}}$ represented in \autoref{fig:not-closed-kleene} is not weakly acyclic. 

