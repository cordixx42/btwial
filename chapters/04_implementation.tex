\chapter{Further Implementations}\label{chapter:implementation}

%The main purpose of the thesis is to use weakly acyclic languages to address the coverability problem in Petri nets. Therefore, the concepts from \autoref{chapter:weakly_acyclic_language} are used to adapt the Backwards Reachability Algorithm in Algorithm \autoref{alg:bw}.
\par
%The Algorithm \autoref{alg:bw} is working with infinite upward-closed sets of markings, like $\mathcal{M}$ and $\mathcal{M}_{old}$. Since automata are characterizing languages, which are a set of words, they are a way to describe these infinite sets in a compact finite representation. In fact, for the representation of the upward closed markings, the subset of weakly acyclic languages are sufficient.
\par
%Therefore, the idea is to represent each of the upward-closed sets, which are used in Algorithm \autoref{alg:bw}, by a weakly acyclic DFA. To manage all those DFAs efficiently, a data structure $table$ is introduced. The $table$ stores a set of weakly acyclic DFAs using the concept of the master automaton in \autoref{chapter:weakly_acyclic_language}. It further provides the required set operations $union$ and $intersection$, which calculate the union/intersection of two weakly acyclic languages. Furthermore, there are encoding functions, which take a marking, convert it into a weakly acyclic DFA, and adds it to the $table$.
\par
%For the main operation $pre$, to calculate all possible predecessor markings, the structure of the petri net $N$ comes into use. 


%The given input are the petri net $N$, a start marking $M_{0}$ and an end marking $M$ analogous to Algorithm \autoref{alg:bw}. 
%This Algorithm \autoref{alg:bw_wwa} adds a $table$ data structure, which is representing a set of weakly acyclic automata. Those weakly acyclic languages are corresponding to upward closed sets of markings.
%The $table$ is providing most of the important functions, like $union$ and $pre$. 
%Furthermore, there is $transducer$, a data structure representing one weakly acyclic transducer. With the function $net2transducer$, the information about the flow relation of the transitions in petri net $N$ are given to $transducer$.

\section{Visualization of Data Structures}
We implemented methods for visualizing the Table of Nodes and Transducers. Automatically generated dot files.

\todo{add figures of generated images}

\section{Parser}
\todo{recursive descend parser, stack overflow }
To be able to use our Backwards Reachability Algorithm on existing problem instances, we implement a parser to transform Petri nets of the MIST format into the representation of Petri nets used in our implementation. 
The MIST input format is shown in Figure. 

We use the stucture of a recursive descend parser. 

\section{CorrectnessCheck}
To verify the correctness of \autoref{alg:bw_wwa_final} we implement another procedure $\mm$. For a node $q$ in the table, $\mm$ computes the set of minimal markings for the upward closed set $q$ represents. By checking the correctness of the minimal markings we recognize if \autoref{alg:bw_wwa_final} returns the correct result.

\begin{lemma}\label{lem:sanity}
Let $q$ be a node in the Table of Nodes, which is encoding the upwards closed set $\mathcal{M}$. Then for every minimal marking of $\mathcal{M}$ there exists a path without self loop in the weakly acyclic automaton for $q$.
\end{lemma}

Using Lemma~\autoref{lem:sanity}.
We iterate over all states in the automaton of $q$ and 

\begin{algorithm}[htb]
\caption{Computation of the Minimal Markings}\label{alg:sanity}
\begin{algorithmic}[1]
\Procedure{minMarkings}{$q$}
%\If {$\cache[q]$}
%\Return {$\cache[q]$}
%\EndIf
\State topology 
\State $W \gets (q,$
\EndProcedure
\end{algorithmic}
\end{algorithm}



