\chapter{Preliminaries}\label{chapter:preliminaries}
This chapter aims to define a standardized formal notation of the theoretical concepts, which are used throughout the thesis.

%\section{TODOS}
%\begin{itemize}
%\item include definition of a transducer 
%\item def of upward closed sets / markings
%\item residual $L^{a}$
%\item reachability
%\item power set construction ?
%\item quotient automaton
%\item equivalence relation
%\todo{Partial Order?}
%\end{itemize}

\section{Formal Languages}

\subsubsection{Residual Languages}

The \emph{residual language} of the language $L$ with respect to a letter $a$ is defined as \break
$L^{a} = \{ w \in \Sigma^{*} : aw \in L \}$.

We can extend this definition to the residual language with respect to a word $v \in \Sigma^{*}$ to \break
$L^{v} = \{ w \in \Sigma^{*} : vw \in L \}$.

\subsubsection{Language of an Automaton State}

For a state $q$ in an automaton $A$, we define the language $L_{A}(q)$, which contains all words accepted by $A$ with initial state $q$.

\subsubsection{Transducer}

A \emph{transducer} $\mathcal{T}$ over $\Sigma$ is an NFA over the alphabet $\Sigma \times \Sigma$. $\mathcal{T}$ accepts the pair of words $(w,v)$ with $w = w_{1}w_{2}...w_{n}$ and $v=v_{1}v_{2}...v_{n}$, if it accepts $(w_{1},v_{1})(w_{2},v_{2})...(w_{n},v_{n})$.
We call the set of all $w$ the preimage and the set of all $v$ the image of transducer $\mathcal{T}$.


\section{Petri nets}

A \emph{Petri net} $N$ is a tuple $(P,\Tr,F)$ where 
\begin{itemize}[-,noitemsep]
	\item $P$ is a finite set of \emph{places}
	\item $\Tr$ is a finite set of \emph{transitions} disjoint from $S$
	\item $F \subseteq (P \times \Tr \times \mathbb{N}) \cup (\Tr \times P \times \mathbb{N})$ is a \emph{flow relation}
\end{itemize}

In drawings of Petri nets, the places are usually represented by circles, the transitions by rectangles and the flow relation is shown by numbered arrows between places and transitions.

\subsubsection{Marking}
A \emph{marking} of a Petri net is a mapping $M : P \rightarrow \mathbb{N}$ from the places of the net to the natural numbers, assigning each place a certain amount of tokens. In drawings, the markings are commonly represented by black dots inside the places.


\subsubsection{Firing Rule}
A transition $t \in \Tr$ is \emph{enabled} at a marking $M$, if for all tuples $(p,t,n) \in F$, $M$ assigns at least $n$ tokens to place $p$.

If a transition $t$ is enabled at marking $M$, it can \emph{fire} leading from $M$ to another marking $M'$, which we also denote by  $M \xrightarrow{t} M'$. 
The resulting $M'$ is of the following form:
\begin{equation*}
M'(p) =  
\begin{cases}
\mathrlap{M(p) + (m - n)}\hphantom{M(p) + (m - n) + 1} \text{ if } \exists (p,t,n) \in F \text{ and } \exists (t,p,m) \in F \\
\mathrlap{M(p) - n}\hphantom{M(p) + (m - n) + 1} \text{ if } \exists (p,t,n) \in F \text{ and } \nexists (t,p,m) \in F \\
\mathrlap{M(p) + m}\hphantom{M(p) + (m - n) + 1} \text{ if }  \nexists (p,t,n) \in F \text{ and } \exists (t,p,m) \in F  \\
\mathrlap{M(p)}\hphantom{M(p) + (m - n) + 1} \text{ otherwise}
\end{cases}
\end{equation*}

\subsubsection{Reachable Markings}
A sequence of transitions $\sigma = t_{1} \dots t_{n}$ is enabled at a marking $M$, if there are markings $M_{1},M_{2},\dots,M_{n}$ such that $M \xrightarrow{t_{1}} M_{1} \xrightarrow{t_{2}} M_{2} \xrightarrow{t_{3}} \dots \xrightarrow{t_{n}} M_{n}$. We denote this by $M \xrightarrow{\sigma} M_{n}$. 

If $M \xrightarrow{\sigma} M'$ for some markings $M,M'$ and some sequence $\sigma$, we say that $M'$ is \emph{reachable} from $M$ and denote this by $M \xrightarrow{*} M'$.


\subsubsection{Upward closed sets of Markings}
For markings $M$ and $M'$, we write $M \ge M'$, if for every place $p$ of a Petri net $M(p) \ge M'(p)$. This means that $M$ assigns at least as many tokens to every state as $M'$. We say that marking $M$ \emph{covers} marking $M'$.


A set $\mathcal{M}$ of markings of the Petri net $N$ is \emph{upward closed} if $M \in \mathcal{M}$ and $M' \ge M$ imply $M' \in \mathcal{M}$. In other words, if a marking exists in the upward closed set, then all other markings, which cover this one marking are also part of the set.

A marking $M$ of an upward closed set $\mathcal{M}$ is minimal, if $\nexists M' \in \mathcal{M}$ such that $M' \le M$. Every upward closed set $\mathcal{M}$ can be uniquely identified by a finite set of minimal elements. 
%Furthermore, every $\mathcal{M}$ has finitely many minimal elements.