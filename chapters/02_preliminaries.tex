\chapter{Preliminaries}\label{chapter:preliminaries}
This chapter aims to define a standardized formal notation of the theoretical concepts used throughout the thesis.

%\section{TODOS}
%\begin{itemize}
%\item include definition of a transducer 
%\item def of upward closed sets / markings
%\item residual $L^{a}$
%\item reachability
%\item power set construction ?
%\item quotient automaton
%\item equivalence relation
%\todo{Partial Order?}
%\end{itemize}

\section{Formal Languages}

\subsubsection{DFA}
A \emph{deterministic finite automaton} (DFA) is a tuple $A = (Q, \Sigma, q_{0},\delta,F)$ where
\begin{itemize}[-,noitemsep]
\item $Q$ is a finite set of states;
\item $\Sigma$ is a finite alphabet;
\item $q_{0} \in Q$ is a start state;
\item $\delta : Q \times \Sigma \rightarrow Q$ is a transition function;
\item $F \subseteq Q$ is a set of accepting final states.
\end{itemize}

\subsubsection{NFA}
A \emph{nondeterministic finite automaton} (NFA) is a tuple $B = (Q, \Sigma, q_{0},\delta,F)$ where all elements are equivalently defined as for a DFA except the transition function:
\begin{itemize}[-,noitemsep]
\item $\delta : Q \times \Sigma \rightarrow \mathcal{P}(Q)$ where $\mathcal{P}(Q)$ denotes the power set of all states $Q$.
\end{itemize}

\subsubsection{Residual Languages}

The \emph{residual language} of the language $L$ with respect to a letter $a$ is defined as \break
$L^{a} = \{ w \in \Sigma^{*} : aw \in L \}$.

We can extend this definition to the residual language with respect to a word $v \in \Sigma^{*}$ to \break
$L^{v} = \{ w \in \Sigma^{*} : vw \in L \}$.

\subsubsection{Language of an Automaton State}

For a state $q$ in an automaton $A$, we define the language $L_{A}(q)$, which contains all words accepted by $A$ with initial state $q$.

\subsubsection{Transducer}

A \emph{transducer} $\mathcal{T}$ over $\Sigma$ is an NFA over the alphabet $\Sigma \times \Sigma$. 
It is a tuple $\mathcal{T} = (Q,\Sigma,q_{0},\delta,F)$ where 
\begin{itemize}[-,noitemsep]
\item $Q$ is a finite set of states;
\item $\Sigma$ is a finite alphabet;
\item $q_{0} \in Q$ is a start state;
\item $\delta : Q \times (\Sigma \times \Sigma) \rightarrow \mathcal{P}(Q)$ is a transition function;
\item $F \subseteq Q$ is a set of accepting final states.
\end{itemize}
$\mathcal{T}$ accepts the pair of words $(w,v)$ with $w = w_{1}w_{2} \ldots w_{n}$ and $v=v_{1}v_{2} \ldots v_{n}$, if it accepts $(w_{1},v_{1})(w_{2},v_{2}) \ldots (w_{n},v_{n})$.
We call the set of all $w$ the preimage and the set of all $v$ the image of transducer $\mathcal{T}$.


\section{Petri nets}

A \emph{Petri net} $N$ is a tuple $(P,\Tr,F)$ where 
\begin{itemize}[-,noitemsep]
	\item $P$ is a finite set of \emph{places};
	\item $\Tr$ is a finite set of \emph{transitions} disjoint from $S$;
	\item $F \subseteq (P \times \Tr \times \mathbb{N}) \cup (\Tr \times P \times \mathbb{N})$ is a \emph{flow relation}.
\end{itemize}

In drawings of Petri nets, the places are usually represented by circles, the transitions by rectangles and the flow relation is shown by numbered arrows between places and transitions. \autoref{fig:petex} shows an example with three places ($p_{1},p_{2},p_{3}$) and two transitions ($t_{1},t_{2}$). For this instance the flow relation contains the following elements: $(p_{1},t_{1},2), (t_{1},p_{2},1), (t_{1},p_{3},2),\allowbreak (p_{2},t_{2},2), (p_{3},t_{2},1)$, and $(t_{2},p_{1},3)$.

\begin{figure}[H]
\centering 
\caption{Example of a Petri Net}\label{fig:petex}
\begin{tikzpicture}[>={Stealth[round]}]
% Place 1
\node[place,
    tokens=1,
    label=$p_1$] (s1) at (0,2) {};
 
% Place 2
\node[place,
    tokens=3,
    label=$p_2$] (s2) at (4,3) {};
 
% Place 3
\node[place,
    tokens=2,
    label=below:$p_3$] (s3) at (4,1) {};

\node[transition, label=$t_1$](t1) at (2,2) {};
\node[transition, label=$t_2$](t2) at (6,2) {};

		\draw   (s1) edge[above] node{$2$} (t1);
		\draw   (t1) edge[above] node{$1$} (s2);
		\draw   (t1) edge[below] node{$2$} (s3);
		\draw   (s2) edge[above] node{$2$} (t2);
		\draw   (s3) edge[below] node{$1$} (t2);
		\draw   (t2) edge[bend left=90, below] node{$3$} (s1);
		
\end{tikzpicture}
\end{figure}

\subsubsection{Marking}
A \emph{marking} of a Petri net is a mapping $M : P \rightarrow \mathbb{N}$ from the places of the net to the natural numbers, assigning each place a certain amount of tokens. In drawings, the markings are commonly represented by black dots inside the places like in \autoref{fig:petex}.


\subsubsection{Firing Rule}
A transition $t \in \Tr$ is \emph{enabled} at a marking $M$, if for all tuples $(p,t,n) \in F$, $M$ assigns at least $n$ tokens to place $p$.

If a transition $t$ is enabled at marking $M$, it can \emph{fire} leading from $M$ to another marking $M'$, which we also denote by  $M \xrightarrow{t} M'$. 
The resulting $M'$ is of the following form:
\begin{equation*}
M'(p) =  
\begin{cases}
\mathrlap{M(p) + (m - n)}\hphantom{M(p) + (m - n) + 1} \text{ if } \exists (p,t,n) \in F \text{ and } \exists (t,p,m) \in F \\
\mathrlap{M(p) - n}\hphantom{M(p) + (m - n) + 1} \text{ if } \exists (p,t,n) \in F \text{ and } \nexists (t,p,m) \in F \\
\mathrlap{M(p) + m}\hphantom{M(p) + (m - n) + 1} \text{ if }  \nexists (p,t,n) \in F \text{ and } \exists (t,p,m) \in F  \\
\mathrlap{M(p)}\hphantom{M(p) + (m - n) + 1} \text{ otherwise}
\end{cases}
\end{equation*}

\subsubsection{Reachable Markings}
A sequence of transitions $\sigma = t_{1} \dots t_{n}$ is enabled at a marking $M$, if there are markings $M_{1},M_{2},\dots,M_{n}$ such that $M \xrightarrow{t_{1}} M_{1} \xrightarrow{t_{2}} M_{2} \xrightarrow{t_{3}} \dots \xrightarrow{t_{n}} M_{n}$. We denote this by $M \xrightarrow{\sigma} M_{n}$. 

If $M \xrightarrow{\sigma} M'$ for some markings $M,M'$ and some sequence $\sigma$, we say that $M'$ is \emph{reachable} from $M$ and denote this by $M \xrightarrow{*} M'$.


\subsubsection{Upward Closed Set of Markings}
For markings $M$ and $M'$, we write $M \ge M'$, if for every place $p$ of a Petri net $M(p) \ge M'(p)$. This means that $M$ assigns at least as many tokens to every state as $M'$. We say that marking $M$ \emph{covers} marking $M'$.


A set $\mathcal{M}$ of markings of the Petri net $N$ is \emph{upward closed} if $M \in \mathcal{M}$ and $M' \ge M$ imply $M' \in \mathcal{M}$. In other words, if a marking exists in the upward closed set, then all other markings, which cover this one marking are also part of the set.

A marking $M$ of an upward closed set $\mathcal{M}$ is minimal, if $\nexists M' \in \mathcal{M}$ such that $M' \le M$. Every upward closed set $\mathcal{M}$ can be uniquely identified by a finite set of minimal elements. 
%Furthermore, every $\mathcal{M}$ has finitely many minimal elements.