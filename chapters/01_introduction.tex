\chapter{Introduction}\label{chapter:introduction}

%The main purpose of the thesis is to use a very recently studied class of formal languages, the weakly acyclic languages, to address the coverability problem in Petri nets. 
%
%jonas war hier

Petri nets are a useful mathematical model to describe distributed systems. Their generality and abstractism makes them applicable in a wide range of areas, including the verification of concurrent programs \cite{german_92,kaiser_14}, manufacturing and control systems \cite{silva_97, wenzelburger_19}, biological systems \cite{carvalho_18, cherdal_18} and business processes \cite{hee_13}.


This formal model consists of places, which are connected by transitions. The places contain tokens, which are able to move between the places by executing transitions. We call a marking a configuration of the Petri net, where each place is assigned a specific amount of tokens.
This formal model allows to examine interesting properties of real-world systems on an abstract mathematical level.

Some most studied decision problems in this area are about reachability and boundedness of Petri nets. The reachability problem investigates, which markings are reachable from a given start marking by executing transitions of the net. The boundedness problem observes, if the amount of tokens in the places are limited by an upper bound for all reachable markings.

In this thesis, we shift the attention to another important decision problem, the coverability problem of Petri nets, which investigates the following issue:
Can a given start marking reach a marking larger than a given target marking, by executing an arbitrary amount of transitions of the Petri net? Here, a larger marking means, that this marking assigns at least as many tokens to the place compared to another.

\todo{why coverability useful}

There exist various decision procedures for the coverability problem. For instance, the coverability graph algorithm builds the coverability graph of a given Petri net and start marking, which can be efficiently checked for coverability given any target marking \cite{finkel_91}. Rackoff's algorithm is another procedure, which makes use of the reachability graph by constructing it until a specific depth dependent on the target marking \cite{rackoff_78,esparza_19}. 

The decision procedure we are examining for the thesis is the Backwards Reachability Algorithm. This algorithm starts with the infinite set of all markings, which are larger than the target marking. Such a set, where if a certain element is included, all elements greater than it are also included, we call upward closed. By iteratively computing and assembling all possible predecessor markings of this upward closed set and checking if the predecessors contain the start marking, we obtain the solution to the coverability problem \cite{abdulla_96}. 

To represent the infinite upward closed sets of the Backwards Reachability Algorithm, common approaches operate with the unique finite set of minimal markings, which every upward closed set possesses. However, the efficiency of this algorithm depends heavily on the number of minimal markings for the upward closed sets, which might grow exponentially with the number of places in the Petri net \cite{delzanno_00}. To address this issue, studies have been conducted on data structures designed for representing infinite sets \cite{wolper_98, moller_99, delzanno_04}.

%The algorithm computes all possible predecessor markings of the upward closed set of the given target marking. This upward closed set is an infinite set containing all markings, which are assigning at least as many tokens to each place as the target marking. Taking a look back at the coverability problem, this set contains all markings larger than the target marking. Therefore, by calculating all predecessors for this set
%
% start with the  In each iteration we execute a transition backwards and accumulate all the predecessor markings. Finally we can check if


%describe backwards reachability algorithm loosely, introduce upward closed sets

In this thesis we want to investigate a novel approach to the Backwards Reachability Algorithm by representing the infinite upward closed sets with formal languages. Using deterministic finite automata for the 
We make the following contributions:

\begin{itemize}[-]
\setlength\itemsep{0.2em}
\item introduce the class of weakly acyclic languages, which are a subset of regular languages (\autoref{chapter:weakly_acyclic_language})
\item design a data structure Table of Nodes for representing weakly acyclic automata efficiently (\autoref{chapter:datastructure})
\item implement important operations on this data structures (\autoref{chapter:datastructure})
\item define a bidirectional mapping between weakly acyclic automata and Petri net configurations (\autoref{chapter:coverability})
\item construct a bidirectional mapping between a transducer - an automaton with input and output - and a Petri net (\autoref{chapter:coverability})
\item integrate the Table of Nodes into the Backwards Reachability Algorithm
\item perform benchmarks on our implementation and compare it to other coverability tools (\autoref{chapter:evaluation})
\end{itemize}

The benchmarks show, that our approach can characterize the infinite sets of markings compactly in the data structure we developed. Furthermore, comparisons with existing tools emphasize, that our implementation is able to compete with 